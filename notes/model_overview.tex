\documentclass{article}
\usepackage[utf8]{inputenc}
\usepackage{amsmath,amsfonts,amssymb,enumerate}
\usepackage{graphicx}
\usepackage{algorithm}
\usepackage{algorithmic}

\title{SkellySim: Renovated Cell Simulator}
\author{G\"{o}kberk Kabacao\u{g}lu and Robert Blackwell}
\date{April 2021}

\newcommand{\OOmega}{\boldsymbol{\Omega}}
\newcommand{\uu}{\mathbf{u}}
\newcommand{\UU}{\mathbf{U}}
\newcommand{\XX}{\mathbf{X}}
\newcommand{\RR}{\mathbf{R}}
\newcommand{\TT}{\mathbf{T}}
\newcommand{\nn}{\mathbf{n}}
\newcommand{\xx}{\mathbf{x}}
\newcommand{\yy}{\mathbf{y}}
\newcommand{\qq}{\mathbf{q}}
\newcommand{\rr}{\mathbf{r}}
\newcommand{\ee}{\mathbf{e}}
\newcommand{\rrhat}{\hat{\mathbf{r}}}
\newcommand{\ubarF}{\bar{\mathbf{u}}^F}
\newcommand{\ubarFp}{\bar{\mathbf{u}}^{F,+}}
\newcommand{\uF}{{\mathbf{u}}^F}
\newcommand{\ubarP}{\bar{\mathbf{u}}^P}
\newcommand{\ubarPp}{\bar{\mathbf{u}}^{P,+}}
\newcommand{\uP}{{\mathbf{u}}^P}
\newcommand{\ubarG}{\bar{\mathbf{u}}^{\Gamma_0}}
\newcommand{\ubarGp}{\bar{\mathbf{u}}^{\Gamma_0,+}}
\newcommand{\uG}{{\mathbf{u}}^{\Gamma_0}}
\newcommand{\ff}{\mathbf{f}}
\newcommand{\FF}{\mathbf{F}}
\newcommand{\GG}{\mathbf{G}}
\newcommand{\MM}{\mathbf{M}}
\newcommand{\calG}{\mathcal{G}}
\newcommand{\calN}{\mathcal{N}}
\newcommand{\calT}{\mathcal{T}}
\newcommand{\VV}{\mathbf{V}}
\newcommand{\DD}{\mathbf{D}}
\newcommand{\II}{\mathbf{I}}
\newcommand{\LL}{\mathbf{L}}

\begin{document}

\maketitle

\section{Integral equation formulation}
We simulate the flows of elastic filaments and their assemblies with rigid bodies with or without a confining rigid boundary. As the flow is governed by the Stokes equations, we opted for an integral equation formulation of the problem. In this formulation, the fundamental solutions to the Stokes equations, the Stokeslet tensor $\GG$, the Stresslet tensor $\TT$ and the Rotlet tensor $\RR$, are given by
\begin{align}
    \GG(\rr) & = \frac{1}{8\pi \mu} \frac{\II + \rrhat\rrhat}{\| \rr \|}, \\
    \TT(\rr) & = -\frac{3}{4\pi}\frac{\rrhat \rrhat \rrhat}{\| \rr \|^2}, \\
    \RR(\rr)\cdot \LL & = \frac{1}{8 \pi \mu} \frac{\LL \times \rrhat}{\|\rr\|^2},
\end{align}
where $\rrhat := \rr / \|\rr\|$, $\rr = \xx - \yy$ and $\mu$ is the fluid viscosity.

The solution of the Stokes equations is defined as the convolutions of these tensors along fiber centerlines $\gamma$ and surfaces $\Gamma$. The single-layer and the double-layer integrals are, respectively,
\begin{align}
    \calG_{\gamma}[\ff](\xx) & := \int_{\gamma} ds_{y} \GG(\xx-\yy) \cdot \ff(\yy),\\
    \calT_{\Gamma}[\qq](\xx) & := \int_{\Gamma} dS_{y} \nn(\yy) \cdot \TT (\xx-\yy) \cdot \qq(\yy).
\end{align}

The fluid velocity at a point $\xx$ is a superposition of velocities arising from integral contributions from each surface (periphery and immersed objects) and those from fiber centerlines:
\begin{equation}
    \uu(\xx) = \uu^P(\xx) + \uu^F(\xx) + \uu^{\Gamma_0}(\xx) \quad \text{with} \,\, \uu^P = \sum_{n = 1}^{N_P} \uu_n^P, \,\, \uu^F = \sum_{m=1}^{N_F} \uu_m^F.
\end{equation}

\paragraph*{Notation.} We denote all the velocity contributions other than those from the fibers with $\ubarF = \uu - \uF$. Similarly, we use $\ubarP$ for the immersed objects and $\ubarG$ for the periphery.

\subsection{The contribution from the periphery $\Gamma_0$}
We express the fluid flow inside a periphery as a double-layer integral over $\Gamma_0$ with an unknown vector density $\qq_0$:
\begin{align}
    \uG(\xx) & = \calT_{\Gamma_0}[\qq_0](\xx) + \calN_{\Gamma_0}[\qq_0](\xx), \quad \text{where} \\
    \calN_{\Gamma_0}[\qq_0](\xx) & = \int_{\Gamma_0}dS_y [\nn(\xx)\nn(\yy)]\cdot\qq_0(\yy).
\end{align}
We find the density $\qq_0$ by solving a Fredholm integral equation of the second kind:
\begin{equation}
    -\frac{1}{2}\qq_0 + \calT_{\Gamma_0}[\qq_0](\xx) + \calN_{\Gamma_0}(\xx) = -\ubarG.
\end{equation}


\subsection{The contribution from the immersed objects $\Gamma_n$}
We consider the flow generated by the motion of rigid immersed bodies, each moving under an externally imposed force $\FF^{ext}_n$ and torque $\LL^{ext}_n$ and the background flow $\ubarP_n$. Let $\UU^P_n$ and $\OOmega^P_n$ be the particle's unknown translational and rotational velocities, respectively. We denote the particle's center of mass with $\XX_n^P$. The flow generated by the $n^{th}$ rigid body is, then,
\begin{equation}
    \uu_n^P(\xx) = \calT_{\Gamma_n}[\qq_n](\xx) + \GG(\xx-\XX_n^P)\cdot\FF_n^{ext} + \RR(\xx-XX_n^P)\cdot\LL_n^{ext}.
\end{equation}
We find the unknown density $\qq_n$ by solving
\begin{equation}
    \UU_n^p + \OOmega_n^P \times (\xx-\XX_n^P) - \ubarP_n(\xx) = -\frac{1}{2}\qq_n(\xx)+\calT_{\Gamma_n}[\qq_n](\xx) + \GG(\xx-\XX_n^P)\cdot\FF_n^{ext} + \RR(\xx-\XX_n^P)\cdot\LL_n^{ext}
\end{equation}
with the constraints due to the fact that the particle moves as a rigid body, i.e.,
\begin{align}
  \frac{1}{\|\Gamma_n\|}\int_{\Gamma_n}dS_y \qq_n(\yy) & = \uu_n^P,\\
  \frac{1}{\|\Gamma_n\|}\int_{\Gamma_n}dS_y (\yy-\XX_n^p) \times \qq_n(\yy) & = \OOmega_n^P.
\end{align}

\subsection{The contribution from the fibers $\gamma$}
Let $\XX(s,t)$ denote the fiber centerline parameterized in arc-length $s \in [0, L(t)]$ where $L(t)$ is the fiber's length at time $t$. $s = 0$ and $s = L(t)$ are the fiber's minus and plus ends, respectively. The velocity induced by the fiber at a distal location $\xx$ is
\begin{equation}{\label{eq:Gotz}}
\uF (\xx) = \calG[\ff](\xx) + \frac{\varepsilon^2}{2}\mathcal{W}[\ff](\xx).
\end{equation}
Here, the second term on the right hand side is the Stokes Doublet,
\begin{equation*}
    \mathcal{W}[\ff](\xx) = \frac{1}{8\pi\mu}\int_0^{L(t)} ds' \frac{\II-3\rrhat\rrhat}{\|\rr\|^3}\cdot\ff(s'),
\end{equation*}
and it is not included in our simulations.

Let $\VV(s) = \frac{d\XX}{dt}$ be the fiber's centerline velocity, then the slender body theory delivers the self-induced motion of the fiber itself to leading orders in $\varepsilon$ as
\begin{equation}\label{eq:SBT}
    \VV(s) - \ubarF(s) = (\MM\cdot\ff)(s) + \mathcal{K}[\ff](s).
\end{equation}
where
\begin{equation}
    \MM\cdot\ff = \frac{1}{8\pi\mu}[-\ln(\varepsilon^2e)(\II + \XX_s \XX_s) + 2(\II -\XX_s \XX_s)]\cdot \ff.
\end{equation}
Here, $\XX_s$ is unit tangent to the fiber and the operator $\mathcal{K}$ is called finite part integral. Neglecting $\mathcal{K}$ results in the local slender body theory. Refer to (Nazockdast et al. (2016), Journal of Computational Physics) for the details. In our code, we implemented the operator $\mathcal{K}$, however, we do not currently use it in the simulations.

Eq.\eqref{eq:SBT} relates the fiber velocity to the fiber forces acting upon the fluid. Since inertial effects are negligible in the Stokes regime, the sum of all forces at any point along the fiber is identically zero. So, the hydrodynamic force applied from the fiber to the fluid, $\ff$, balances internally generated forces $\ff^I$, arising for instance from elastic deformations of the fibers, and external forces applied to the fiber, $\ff^E$, say by molecular motors carrying payloads, or by gravitational body forces. That is,
\begin{equation}
    \ff = \ff^I + \ff^E.
\end{equation}

\paragraph{Mechanics of elastic fibers.}
The internal elastic forces are related to fiber configurations through appropriate constitutive relations, and here we choose to use the Euler–Bernoulli beam theory for elastic rods. That is, the force per unit length is given as
\begin{equation}
    \ff^I(s) = -E \XX_{ssss} + (T\XX_s)_s,
\end{equation}
where $E$ is the bending stiffness and $T$ is the tension that acts as a Lagrange multiplier to enforce the inextensibility of the fibers. Note that twist elasticity is neglected here. Twist elasticity only becomes relevant when a net torque is applied in the direction of the microtubules/fibers, which only occurs when active forces have a component in the angular direction of the cylindrical fibers.

The inextensibility constraint requires that $s$ is a material parameter and independent of $t$, which implies that $\XX_t(s,t) = V(s,t)$. The constraint is
\begin{equation}\label{eq:inextens}
    \XX_s \cdot \XX_s = 1.
\end{equation}
By differentiating it with respect to time, we obtain the auxiliary constraint:
\begin{equation}\label{eq:auxInextens}
    \VV_s \cdot \XX_s = 0.
\end{equation}
\paragraph*{The auxiliary integro-differential equation for the tension $T$.}
Using the following identitites:
\begin{align}
    \XX_s \cdot \XX_{ss} & = 0, \\
    \XX_s \cdot \XX_{sss} & = -\XX_{ss}\cdot \XX_{ss}, \\
    \XX_s \cdot \XX_{ssss} & = -3\XX_{ss} \cdot \XX_{sss}.
\end{align}
we obtain
\begin{multline}
    -2c_0T_{ss} + (c_0 + c_1)(\XX_{ss}\cdot\XX_{ss})T = (7c_0+c_1)E(\XX_{ss}\cdot\XX_{ssss}) + \\ 6c_0E(\XX_{sss}\cdot\XX_{sss}) + \XX_s \cdot \ubarF_s + 2c_0\XX_s\cdot \ff^E_s + (c_0 - c_1)\XX_{ss}\cdot \ff^E
\end{multline}
where the coefficients are
\begin{align}\label{eq:coeffs}
    c_0 &= \frac{-\ln(\varepsilon^2 e)}{8 \pi \mu}, \\
    c_1 &= \frac{1}{4\pi\mu}.
\end{align}

\subsubsection{Boundary conditions}
The evolution equations are fourth-order in s for X, while Eq.~\eqref{eq:inextens} is second-order in s for T. So, we need three boundary conditions at each end of a fiber.
\paragraph{Free or loaded: Prescribed force and torque.}
Given the external force $\FF^{ext}$ and $\LL^{ext}$, the boundary conditions are
\begin{align}
\FF^{ext} =& -E\XX_{sss} + T\XX_s  \quad  \text{at} \,\, s = L, \\
\LL^{ext} =& -E\XX_{ss} \times \XX_s \quad  \text{at} \,\, s = L, \\
\FF^{ext} =& E\XX_{sss} - T\XX_s \quad  \text{at} \,\, s = 0,\\
\LL^{ext} =& E\XX_{ss} \times \XX_s \quad  \text{at} \,\, s = 0.
\end{align}
The boundary condition for tension is given by taking the inner products of the force boundary conditions with $\XX_s$
\begin{align}
T &= \XX_s \cdot \FF^{ext} - E \XX_{ss} \cdot \XX_{ss} \quad \text{at} \,\, s = L, \\
T &= -\XX_s \cdot \FF^{ext} - E \XX_{ss} \cdot \XX_{ss} \quad \text{at} \,\, s = 0.
\end{align}

\paragraph{Clamped: Prescribed velocity and angular velocity.}
Suppose that the fiber is clamped at its minus end to a rigid body moving with a translational velocity $\UU$ and angular velocity $\OOmega$. Then, the fiber's boundary conditions at $s = 0$ are
\begin{align}
\frac{\partial \XX}{\partial t}  & = \UU + \OOmega \times \mathbf{r}, \label{eq:velocityBC} \\
\frac{\partial \XX_s}{\partial t} & = \OOmega \times \mathbf{r}, \label{eq:rotBC}
\end{align}
where $\rr = \XX_{s=0} - \XX^P$ and $\XX^P$ is the center of the body. Taking the inner product of Eq.~\eqref{eq:velocityBC} with $\XX_s$ gives the boundary condition for tension
\begin{multline}
(\UU + \OOmega \times \rr)\cdot \XX_s  = \\
\ubarF \cdot \XX_s + 2c_0T_s + 6Ec_0\XX_{ss} \cdot \XX_{sss} + 2c_0 T(s) + 2c_0 \XX_s \cdot \ff^{ext}.
\end{multline}

The MT also exerts force and torque on the body. These are
\begin{align}
\FF^P & = -\FF^{e}|_{s = 0}  = -E\XX_{sss} + T\XX_s, \\
\LL^P & = -(\LL^{e}|_{s = 0} + \mathbf{r} \times \FF^{e}|_{s = 0}) = -E\XX_{ss} \times \XX_{s} + \mathbf{r} \times (-E\XX_{sss} + T\XX_s).
\end{align}

\paragraph{Hinged: Prescribed velocity and zero torque.}
Let's say the fiber end is moving with a velocity $\UU$ and is torque-free. The boundary conditions are
\begin{align}
\frac{\partial \XX}{\partial t}  & = \UU, \\
0 & = -E\XX_{ss} \times \XX_s.
\end{align}
For the tension, we take the inner product of the first equation above, i.e.,
\begin{equation}
\UU \cdot \XX_s = \ubarF \cdot \XX_s + 2c_0T_s + 6Ec_0\XX_{ss} \cdot \XX_{sss} + 2c_0 T(s) + 2c_0 \XX_s \cdot \ff^{ext}.
\end{equation}

\subsubsection{Microtubule (de)polymerization kinetics}
Let's define a dimensionless parameter $\alpha(s,t) = 2s/L(t) - 1$ and write $\XX(\alpha, t) = \XX(s(\alpha),t)$. This gives $\alpha_s = 2/L$ and $(\cdot)_s = 2(\cdot)_{\alpha}/L$. The chain rule gives
\begin{equation}
    \frac{\partial \XX(\alpha,t)}{t} = V + \alpha_t \XX_{\alpha} = \VV - \frac{(\alpha+1)\dot{L}}{L}\XX_{\alpha},
\end{equation}
where $\dot{L}$ is the rate of (de)polymerization. Eq.~\ref{eq:auxInextens} for tension can be rewritten with respect to $\alpha$ using the chain rule
\begin{equation}
    \XX_{t\alpha}\cdot\XX_{\alpha} = 0.
\end{equation}

\paragraph*{Our dynamic instability implementation.}
There are five global dynamic instability parameters: the nucleation rate $\dot{n}$ ($s^{-1}$), the catastrophe rate $r_{cat}$ ($s^{-1}$), the growth velocity $\dot{L}$ ($\mu m s^{-1}$), the cortical catastrophe rate $r_{cat}^{\Gamma_0}$ ($s^{-1}$) and the cortical growth velocity $\dot{L}^{\Gamma_0}$ ($\mu m s^{-1}$).

Given the values of those parameters, we calculate the average number of fibers on the body, i.e., $\bar{N}_{f} = \dot{n}/r_{cat}$ and generate $10\times \bar{N}_f$ nucleating sites on the body at the onset.

At every time step, we first check whether a fiber is in the close proximity of the periphery, and if so, we flag the fiber. The dynamic instability immediately follows this step. See the Algorithm 2 for our implementation of the dynamic instability.
\begin{algorithm}[H]
\begin{algorithmic}
 \STATE // Loop over fibers
 \FOR{fiber in fibers}
 \STATE // Set fiber's growth velocity and catastrophe rate
 \IF{fiber's $+$ end is closer to the periphery than $0.75\mu m$}
 \STATE $v_g, r_c = \dot{L}^{\Gamma_0}, r_{cat}^{\Gamma_0}$
 \ELSE
 \STATE $v_g, r_c = \dot{L}, r_{cat}$
 \ENDIF
 \STATE // Choose a random number $r$ from a uniform distribution in $[0, 1]$
 \STATE $r = \mathtt{rand}()$
 \IF{$r > \exp(-\Delta t r_c) $}
 \STATE // Go through catastrophe
 \STATE $\mathtt{remove}$(fiber) and $\mathtt{empty}$(its nucleation site)
 \ELSE
 \STATE // Grow fiber
 \STATE fiber.$L$ += $v_g * \Delta t$
 \STATE fiber.$v_g = v_g$
 \ENDIF

 \ENDFOR

 \STATE // Loop over bodies and nucleate fibers on them
 \FOR{body in bodies}
 \STATE // Find the number of fibers to be nucleated
 \STATE $N_{fibers} = \mathtt{poisson}(\dot{n} \Delta t)$
 \STATE // Randomly pick $N_{fibers}$ nucleating sites among the empty ones
 \STATE site\_xyz = body.empty\_site\_xyz($\mathtt{randint}([1, \mathtt{length}$(empty\_site\_xyz)$], N_{fibers})$)
 \FOR{xyz in site\_xyz}
 \STATE // At each sampled site, create a straight fiber along the normal direction of the surface with a minimum fiber length
 \STATE $s = \mathtt{linspace(0, 2, N_p)}$
 \STATE $x, y, z = n_x s, n_y s, n_z s$
 \STATE $X := [x\,\, y\,\, z]$
 \STATE // Finally the fiber configuration is:
 \STATE $X = X L_{\min}/2 + xyz$
 \STATE // Update the list of the body's empty sites
 \ENDFOR
 \ENDFOR
\end{algorithmic}
\caption{$P = \mathtt{dynamicInstability}$(fibers, bodies, time step size $\Delta t$)}
\end{algorithm}



\subsection{Formulation summary}
Following the formulation summary section in (Nazockdast et al. (2016), JCP), here we summarize the formulation in the context of the biophysical problems and consider only one immersed rigid body(i.e., $N_P$ = 1 with surface $P$) and many fibers all attached at their minus-ends ($s = 0$) to that body. The primary unknowns of the system are the double-layer densities $\qq_0$ and $\qq_1$ on the periphery and the rigid immersed body respectively, the translational and angular velocities of the immersed body, $\UU^P$ and $\OOmega^P$ , their associated external forces and torques (due to the attachment of the fibers to the body), $\FF^{ext}_{body}$ and $\LL^{ext}_{body}$, and velocities $\VV_m$ and tensions $T_m$ of the fibers $(m = 1, \dots, N_F)$. The equations we solve given proper constraints and boundary conditions are below.
The principal equations are
\begin{itemize}
    \item \textbf{Periphery}:
    \begin{equation}
     -\frac{1}{2}\qq_0(\xx) + \calT_{\Gamma_0}[\qq_0](\xx) + \calN_{\Gamma_0}[\qq_0](\xx) + \ubarG(\xx) = 0.
    \end{equation}

    \item \textbf{Immersed body}:
    \begin{multline}
     -\frac{1}{2}\qq_1(\xx') + \calT_P [\qq_1](\xx') + \GG\cdot\FF^{ext}_P + \RR\cdot\LL^{ext}_P + \ubarP(\xx') \\
     = \UU^P + \OOmega^P \times (\xx' - \XX^P).
    \end{multline}

    \item \textbf{Fibers}:
    \begin{equation}
      (\MM\cdot\ff_m)(\alpha) + \ubarF_m (\alpha) = \frac{\partial \XX_m(\alpha,t)}{\partial t} + \frac{(\alpha +1)\dot{L}}{L} (\XX_m)_{\alpha}.
    \end{equation}
\end{itemize}

Here $\xx \in \Gamma_0$, $\xx' \in P$ and $\alpha \in [-1, 1]$ and the equation for fibers is repeated for $m = 1, \dots, N_F$. The complementary velocities are
\begin{align}
    \ubarG(\xx) & = \uP(\xx) + \sum_{m=1}^{N_F}\uF_m(\xx),\\
    \ubarP(\xx') & = \uG(\xx') + \sum_{m=1}^{N_F}\uF_m(\xx'),\\
    \ubarF_m(\alpha) &= \uG(\XX_m(\alpha)) + \uP(\XX_m(\alpha)) + \sum_{l = 1, l\neq m}^{N_F} \uF_l(\XX_m(\alpha)),
\end{align}
where
\begin{align}
    \uG(\xx) & = \calT_{\Gamma_0}[\qq_0](\xx),\\
    \uP(\xx) & = \calT_P[\qq_1](\xx) + \GG(\xx-\XX^P)\cdot\FF^{ext}_P + \RR(\xx-\XX^P)\cdot\LL^{ext}_P,\\
    \uF_m(\xx) &= \calG_m[\ff_m](\xx).
\end{align}

For the principal equations to be a closed set of equations, the system requires $12N_P$ constraints as well as a constitutive law relating the configuration of fibers to their elastic force. We chose the Euler-Bernoulli model subject to local inextensibility constraint as the constitutive model. This choice also requires $14N_F$ constraints for the fibers (four constraints for vector position and two constraints for scalar tension). Those constraints depend on the choice of boundary condition for the fibers (mentioned previously). $6N_P$ constraints are given by the rigid body motion of the immersed body and $6N_P$ constraints are given by the force and torque balance on the particles. In summary those are:
\begin{itemize}
    \item \textbf{Immersed body}:
    \begin{equation}
      \frac{1}{\|P\|}\int_P dS_y \qq_1(\yy) = \UU^P,
    \end{equation}
    \begin{equation}
      \frac{1}{\|P\|}\int_P dS_y (\yy-\XX^P) \times \qq_1(\yy) = \OOmega^P,
    \end{equation}
    \item \textbf{Fibers}:
    \begin{equation}
        \ff_m^I = -E(\XX_m)_{ssss} + (T_m(\XX_m)_s)_s,
    \end{equation}
    \begin{equation}
        (\XX_m)_{t\alpha} \cdot (\XX_m)_{\alpha} = 0,
    \end{equation}
    \item \textbf{Fiber-body attachment}:
    \begin{align}
        \FF^{ext}_P & = -\sum_{m = 1}^{N_F} \FF_m^{ext},\\
        \LL^{ext}_P & = - \sum_{m = 1}^{N_F} \left(\LL_m^{ext} + (\XX_m|_{s=0} - \XX^P)\times \FF_m^{ext} \right).
    \end{align}
\end{itemize}


\section{Numerical methods}
In this section, we will discuss the spatial discretization of surfaces of the rigid bodies and the outer boundary and evaluating the related integral equations on these surfaces. We then will present the spatial discretization of the centerline of a fiber in the $\alpha$ coordinate to evaluate the required high-order derivatives and integrals with respect to $\alpha$. After discussing the time discretization scheme, we will conclude the section with the resulting linear system of equations and the preconditioner used to accelerate the solve.

\subsection{Spatial discretization}
We solve the boundary integral equations numerically using the Nystr\"{o}m method. For the immersed bodies and periphery, we use a uniform grid in the polar and azimuthal angles to triangulate the surfaces. The quadrature weights come from a technique motivated by radial basis function-generated finite differences (RBF-FD). See J. A. Reeger, B. Fornberg, and M. L. Watts 2016, Numerical quadrature over smooth, closed surfaces, \textit{Proc. R. Soc. A} 472: 20160401). We represent fiber centerline positions and tensions by uniformly distributed points in $\alpha$, and use trapezoidal rule to compute integrals along the fiber centerlines.

\paragraph*{Remark:} In the case of the fiber-body attachement, the body's surface where the fibers are attached is away from the triangulated surface where the quadratures are computed. That is, we define two radii for a spherical rigid body: 1) quadrature radius, 2) geometric radius at which the fibers are attached. The quadrature radius is 80\% of the geometric radius.



\subsubsection{Singular integrals over surfaces}
We use singularity subtraction for the evaluation of the singular integral $\calT$, i.e.,
\begin{align}
    -\frac{1}{2}\qq_0 + \calT_{\Gamma_0}[\qq_0](\xx) = \calT_{\Gamma_0}[\qq_0 - \qq_0(\xx)] = \calT_{\Gamma_0} - \sum_{i=1}^3 \left(\calT_{\Gamma_0}[\ee_i](\xx)\right)\qq_{0,i}(\xx).
\end{align}
Here, $\ee_i$ is the unit vector in the $i^{th}$ direction.


\subsubsection{Fiber representation}
For fiber representation, we abandoned the Chebyshev basis because we have not observed a spectral convergence in test problems. The simplest test problem involved two fibers attached to a body at (0,0,r) and (0,0,-r) respectively. Then, the body is pushed in $y$-direction. The error in the body's final position decreases from $10^{-2}$ to $10^{-5}$ as the resolution increases from 12 to 48. However, as the resolution increases further, the error starts increasing. Here, the error is measured based on the next finer resolution. In order to resolve that issue, we represented the fiber uniformly distributed points in $\alpha$ and used the $4^{th}$-order finite differences to compute the derivatives.

\subsubsection{Integration over fiber centerlines}
Since the collocation points are uniformly distributed along the fiber centerline, we use the trapezoidal rule for the quadratures.

\subsubsection{Evaluation of nearly singular integrals}
So far, we have not implemented any special quadrature for nearly singular integrals. Instead, we naively regularize the integrals by adding a small number to the denominator.

\subsection{Temporal discretization}
We use a variation of the implicit-explicit method where Eq.~\eqref{eq:SBT} is linearized and the numericall stiff terms (e.g., bending force) are treated implicitly. The linearization is done by computing the geometric properties of surfaces and fibers’ centerlines (e.g., tangent vector, Jacobian, collocation points, etc.) explicitly and treating the forces and densities defined on them implicitly.

We use the $"+"$ superscript to mark the unknowns to be determined at the next time-step. The unmarked variables are calculated at the current time-step. Discretizing the evolution equations, using the backward Euler method, we have

\begin{itemize}
    \item \textbf{Periphery}:
    \begin{equation}
     -\frac{1}{2}\qq^+_0(\xx) + \calT_{\Gamma_0}[\qq_0^+](\xx) + \calN_{\Gamma_0}[\qq_0^+](\xx) + \ubarGp(\xx) = 0.
    \end{equation}

    \item \textbf{Immersed body}:
    \begin{multline}
     -\frac{1}{2}\qq_1^+(\xx') + \calT_P [\qq_1^+](\xx') + \GG\cdot\FF^{ext,+}_{P} + \RR\cdot\LL^{ext,+}_{P} + \ubarPp(\xx') \\
     = \UU^{P,+} + \OOmega^{P,+} \times (\xx' - \XX^P).
    \end{multline}

    \item \textbf{Fibers}:
    \begin{equation}
      (\MM\cdot\ff_m^+)(\alpha) + \ubarFp_m (\alpha) = \frac{\XX_m^+ - \XX_m}{\Delta t} + \frac{(\alpha +1)\dot{L}}{L} (\XX_m)_{\alpha}.
    \end{equation}
\end{itemize}

Be careful that the $\FF_P^{ext,+}$ and $\LL_P^{ext,+}$ are the force and torque due to the fiber-body attachment and treated implicitly. However, the external force and torque such (for example due to an electric field) are treated explicitly. Note that the complementary flow is treated implicitly. The velocities induced by the structures are given as
\begin{align}
    \uu^{\Gamma_0,+}(\xx) & = \calT_{\Gamma_0}[\qq_0^+](\xx),\\
    \uu^{P,+}(\xx) & = \calT_P[\qq_1^+](\xx) + \GG(\xx-\XX^P)\cdot\FF^{ext,+}_P + \RR(\xx-\XX^P)\cdot\LL^{ext,+}_P,\\
    \uu^{F,+}_m(\xx) &= \calG_m[\ff^+_m](\xx).
\end{align}
The force along the fiber is computed as
\begin{equation}
    \ff_m^+ = -E\mathbf{D}_s^4\XX_m^+ + \mathbf{D}_s\left(T_m^+(\XX_m)_s\right),
\end{equation}
where $\mathbf{D}_s$ is the spatial differentiation operator.

The constraints are given as
\begin{itemize}
    \item \textbf{Immersed body}:
    \begin{equation}
      \frac{1}{\|P\|}\int_P dS_y \qq_1^+(\yy) = \UU^{P,+},
    \end{equation}
    \begin{equation}
      \frac{1}{\|P\|}\int_P dS_y (\yy-\XX^P) \times \qq_1^+(\yy) = \OOmega^{P,+},
    \end{equation}
    \item \textbf{Fibers}:
    \begin{equation}
        \left(\mathbf{D}_{\alpha}\XX_m^+\right)\cdot(\XX_m)_{\alpha} = (\XX_m)_{\alpha} \cdot (\XX_m)_{\alpha},
    \end{equation}
    \item \textbf{Fiber-body attachment}:
    \begin{align}
        \FF^{ext,+}_P & = -\sum_{m = 1}^{N_F} \FF_m^{ext,+},\\
        \LL^{ext,+}_P & = - \sum_{m = 1}^{N_F} \left(\LL_m^{ext,+} + (\XX_m|_{s=0} - \XX^P)\times \FF_m^{ext, +} \right).
    \end{align}
\end{itemize}

The boundary conditions are treated implicitly as well and linearized if necessary. The position of the rigid body and the length of each fiber are updated using $\XX^{P,+} = \XX^P + \Delta t \UU^P$ and $L(t^+) = L(t) + \dot{L}(t)\Delta t$, respectively.

Particularly, the discretized fiber equation for $\XX$ is
\begin{multline}
    \frac{\XX^+}{\Delta t} + E \left(c_0(1+\XX_s \XX_s) + c_1(1-\XX_s \XX_s)\right)\XX_{ssss}^+ - \left(2c_0 T_s^+ + (c_0 + c_1) \XX_{ss} T^+\right) \\ + \ubarFp
    = \frac{\XX}{\Delta t} + \dot{L}\frac{\alpha+1}{2}\XX_s.
\end{multline}
The equation for tension $T$ is
\begin{multline}
-2c_0T_{ss}^+ + (c_0 + c_1)(\XX_{ss}\cdot\XX_{ss})T^+ = (7c_0+c_1)E(\XX_{ss}\cdot\XX_{ssss}^+) + \\ 6c_0E(\XX_{sss}\cdot\XX_{sss}^+) + \XX_s \cdot \ubarFp_s + 2c_0\XX_s\cdot \ff^E_s + (c_0 - c_1)\XX_{ss}\cdot \ff^E
\end{multline}

\paragraph*{Penalty term for imposing inextensibility.} We modify the inextensibility condition Eq.~\ref{eq:inextens}:
\begin{equation}
    \VV_s \cdot \XX_s  = \beta (1-\XX_s \cdot \XX_s)
\end{equation}
with $\beta$ the penalty parameter. We discretize the penalty term as $\XX_s \cdot \XX^+_s$.

\subsection{Linear solver and preconditioner}
We solve the system of equations described above iteratively using a preconditioned GMRES method. For the fibers, we use the Jacobi block preconditioning scheme, where the self-interaction blocks are formed and inverted using LU decomposition. For immersed particles and the periphery, we use a Jacobi preconditioner considering only the diagonal elements of the self-interaction blocks.

\end{document}
